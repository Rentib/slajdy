\documentclass[../main.tex]{subfiles}
\usepackage[polish]{babel}

\begin{document}

\section{Silnie spójne składowe (SCC)}

\subsection{Wprowadzenie}

\begin{frame}{\subsecname}{Definicja problemu}

Dany jest skierowany graf $G = \langle V, E \rangle$.
Mogą istnieć w nim pętle i krawędzie wielokrotne.
Silnie spójna składowa to maksymalny podzbiór wierzchołków $C$, taki że między każdą parą jego
wierzchołków istnieje ścieżka ($\forall_{v,u \in C} v \rightsquigarrow u \land u 	\rightsquigarrow v$).\\
\pause
Jest oczywiste, że dwie silnie spójne składowe grafu $G$ są rozłączne, inaczej nie byłyby to
maksymalne podzbiory. Możemy w takim razie stworzyć nowy graf $G^{SCC}$, w którym
wierzchołki to będą silnie spójne składowe $G$. Graf silnie spójnych składowych jest acykliczny,
ponieważ jeśli istniałby cykl, to nie byłyby to maksymalne podgrafy.

\end{frame}

\begin{frame}[fragile]{Przykład}
\begin{tikzpicture}

  \begin{scope}[every node/.style={circle, draw, minimum size = 7.5mm, fill=gray!40}]
    \node (1) at (0,5.5) {1};
    \node (2) at (3,5) {2};
    \node (3) at (4,4) {3};
    \node (4) at (1,3.5) {4};
    \node (5) at (7,5) {5};
    \node (6) at (7,4) {6};
    \node (7) at (4,2) {7};
    \node (8) at (6,2) {8};
    \node (9) at (2,1) {9};
    \node (10) at (4,0) {10};
    \node (11) at (6,0) {11};
    \node (12) at (7,1) {12};
  \end{scope}

  \graph[multi, edge={draw, thick}] {
    (2) -> (1),
    (2) -> [bend left] (3),
    (3) -> [bend left] (2),
    (4) -> (2),
    (5) -> (2),
    (5) -> [bend left]( 6),
    (6) -> (3),
    (6) -> [bend left] (5),
    (7) -> {(3), (9)},
    (8) -> {(6), (7)},
    (9) -> (10),
    (10) -> {(7), (11)},
    (11) -> (12),
    (12) -> (8),
  };

  \pause
  \begin{scope}[every node/.style={draw, dashed, gray, rounded rectangle}]
    \node[fit=(1)]{};
    \node[fit=(4)]{};
    \node[fit=(2)(3)]{};
    \node[fit=(5)(6)]{};
    \node[fit=(7)(8)(9)(10)(11)(12)]{};
  \end{scope}

\end{tikzpicture}
\end{frame}

\subsection{Algorytm Kosaraju}

\begin{frame}{\subsecname}{Obserwacja}
Niech $SCC[G] = \left\{S_1, S_2, ..., S_s\right\}$ będzie rodziną zbiorów wierzchołków grafu $G$
takich, że wierzchołki z jednego zbioru tworzą silnie spójną składową.
Niech $\psi(S_i) = \max\left\{tout(v) | v \in S_i\right\}$.
Niech $A, B \in SCC[G]$. Wtedy
$$
  \forall_{A,B\in SCC[G], A \neq B} \psi(A) < \psi(B) \implies \neg\exists A \rightsquigarrow B.
$$
Niech $G^T$ będzie grafem transponowanym $G$, czyli grafem z odwróconymi zwrotami krawędzi.
Wtedy
$$
  SCC[G] = SCC[G^T].
$$
Dowody obu obserwacji są dość łatwe i pozostawiam je jako ćwiczenie.

\end{frame}

\begin{frame}{\subsecname}

Algorytm zaproponowany przez Kosaraju opiera się w głównej mierze na powyższych obserwacjach i 
wygląda tak:
\begin{enumerate}
  \item wyznaczamy czasy wyjścia wierzchołków $G$;
  \item wywołujemy DFS dla $G^T$ w kolejności malejących czasów wyjścia.
        Wszystkie wierzchołki w jednym drzewie przeszukiwania w głąb należą do
        jednej silnie spójnej składowej.
\end{enumerate}

\end{frame}

\begin{frame}[fragile]{\subsecname}{Wyznaczanie kolejności}

\begin{block}{}
\begin{lstlisting}[language = C++]
vector<int> G[MAX_N], Gt[MAX_N];
bool visited[MAX_N];
vector<int> order;
void dfs1(int v) {
  visited[v] = true;
  for (auto u : G[v])
    if (!visited[u])
      dfs1(u);
  order.emplace_back(v);
}
...
for (int v = 1; v <= n; v++)
  if (!visited[v])
    dfs1(v);
\end{lstlisting}
\end{block}

\end{frame}

\begin{frame}[fragile]{\subsecname}

\begin{block}{}
\begin{lstlisting}[language = C++]
int scc[MAX_N], counter = 0;
void dfs2(int v) {
  scc[v] = counter;
  for (auto u : Gt[v])
    if (scc[u] == 0)
      dfs2(u);
}
...
reverse(order.begin(), order.end());
for (auto v : order) {
  if (!scc[v]) {
    counter++;
    dfs2(v);
  }
}
\end{lstlisting}
\end{block}

\end{frame}
\subsection{Graf silnie spójnych składowych (Condensation graph)}

\begin{frame}[fragile]{\subsecname}

\begin{block}{}
\begin{lstlisting}[language = C++]
vector<int> Gscc[counter + 7];
for (int v = 1; v <= n; v++) {
  for (auto u : G[v])
    Gscc[scc[v]].emplace_back(scc[u]);
}
\end{lstlisting}
\end{block}
\pause
Warto zauważyć, że przy takiej implementacji możliwe jest istnienie bardzo wielu wtórnych krawędzi.
To jednak z reguły nie jest problemem (przynajmniej w zadaniach konkursowych).

\end{frame}

\end{document}
