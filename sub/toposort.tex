\documentclass[../main.tex]{subfiles}
\usepackage[polish]{babel}

\begin{document}

\section{Sortowanie topologiczne (Toposort)}

\subsection{Wprowadzenie}

\begin{frame}{\subsecname}{Definicja problemu}

Dany jest graf skierowany $G = \langle V, E \rangle$.
Należy znaleźć taki porządek wierzchołków, aby każda krawędź prowadziła z wierzchołka
o mniejszym indeksie do wierzchołka z większym.\\
Innymi słowy, chcemy znaleźć taką permutację wierzchołków (porządek topologiczny),
który odpowiada porządkowi zdefiniowanemu przez krawędzie (jeśli istnieje krawędź z wierzchołka $v$ do wierzchołka $u$, to $v$ powinno być przed $u$).

\end{frame}

\begin{frame}[fragile]{\subsecname}{Przykład}

\begin{tikzpicture}

\begin{scope}[every node/.style={circle, draw, minimum size = 7.5mm, fill=gray!40}]
  \node[] (1) at (2,2) {\only<2>{1}};
  \node[] (2) at (4,2) {\only<2>{2}};
  \node[] (3) at (1,0) {\only<2>{4}};
  \node[] (4) at (3,0) {\only<2>{3}};
  \node[] (5) at (5,0) {\only<2>{5}};
\end{scope}

\graph[edge = {draw, thick}] {
  (1) -> { (2), (3), (4) },
  (2) -> { (4), (5) },
  (3) -> {},
  (4) -> { (3) },
  (5) -> {},
};

\end{tikzpicture}

\end{frame}

\begin{frame}{\subsecname}

Warto zauważyć, że może istnieć wiele możliwych porządków topologicznych, lub może też nie
istnieć żaden. Wiele poprawnych porządków topologicznych widać choćby na pokazanym przykładzie,
natomiast porządek topologiczny nie istnieje, gdy w grafie istnieje cykl.

\end{frame}

\subsection{Algorytm}

\begin{frame}[fragile]{\subsecname}

Aby rozwiązać problem, najłatwiej jest użyć przeszukiwania w głąb.
Przypuśćmy, że dany graf jest acykliczny. Będziemy tworzyli listę, na której początek będziemy
dodawali wierzchołki, z których wychodzi DFS.

\pause
\begin{block}{}
\begin{lstlisting}[language = C++]
vector<int> toposort; // odwrotna kolejnosc
void dfs(int v) {
  visited[v] = true;
  for (auto u : G[v]) {
    if (!visited[u])
      dfs(v);
  }
  toposort.emplace_back(v);
}
...
reverse(toposort.begin(), toposort.end());
\end{lstlisting}
\end{block}

\end{frame}

\end{document}
