\documentclass[../main.tex]{subfiles}

\begin{document}

\section{Najniższy wspólny przodek (LCA)}

\subsection{Wprowadzenie}

\begin{frame}{\subsecname}{Definicja problemu}
Niech $G$ będzie drzewem.\\
Dla każdego zapytania typu $(v,u)$ chcemy znaleźć najniższego wspólnego przodka 
wierzchołka $v$ i wierzchołka $u$, czyli taki wierzchołek $w$, że należy on do ścieżki z $v$
do korzenia i ścieżki z $u$ do korzenia. Jeśli istnieje wiele takich wierzchołków, to chcemy
wyznaczyć ten najodleglejszy od korzenia drzewa.
\end{frame}

\begin{frame}{\subsecname}{Przykład}

\begin{tikzpicture}

\begin{scope}[every node/.style={circle, draw, minimum size = 7.5mm, fill=gray!40}]
  \node[alt=<4>{fill=pink}] (1) at (5,6) {1};
  \node[alt=<4-5>{fill=pink}] (2) at (3,4) {2};
  \node[] (3) at (7,4) {3};
  \node[] (4) at (1.5,2) {4};
  \node[alt=<2-5>{fill=magenta}] (5) at (4,2) {5};
  \node[] (6) at (6,2) {6};
  \node[] (7) at (7,2) {7};
  \node[] (8) at (8,2) {8};
  \node[] (9) at (.5,0) {9};
  \node[alt=<2-5>{fill=orange}] (10) at (2.5,0) {10};
\end{scope}

\graph[edge = {draw, thick}] {
  (1) -- { (2), (3) },
  (2) -- { (4), (5) },
  (3) -- { (6), (7), (8) },
  (4) -- { (9), (10) },
};

\only<3-4>\graph[edge = {thick, orange, bend left}] {
  (10) -> (4) -> (2) -> (1),
};

\only<3-4>\graph[edge = {thick, magenta, bend right}] {
  (5) -> (2) -> (1),
};

\end{tikzpicture}

\end{frame}

\subsection{Naiwne podejście}

\begin{frame}[fragile]{\subsecname}
Oczywistym rozwiązaniem byłoby skakanie do ojca z niższego z wierzchołków $v$ i $u$.

\pause

\begin{block}{}
\begin{lstlisting}[language=C++]
  while (v != u) {
    if (glebokosc(v) > glebokosc(u))
      v = ojciec[v];
    else
      u = ojciec[u];
  }
  return v;
\end{lstlisting}
\end{block}

\pause

Takie rozwiązanie jest jednak zbyt wolne i w pesymistycznym przypadku wykonuje
$O(n)$ operacji.

\end{frame}

\begin{frame}[fragile]{\subsecname}
Nasz naiwny algorytm można z łatwością usprawnić.\\
\pause
Nie musimy skakać dwoma wierzchołkami. Wystarczy, że będziemy skakać jednym z nich, dopóki
nie doskoczymy do przodka drugiego wierzchołka.
\pause
\begin{block}{}
\begin{lstlisting}[language=C++]
  while (!jest_przodkiem(v, u))
    v = ojciec[v];
  return v;
\end{lstlisting}
\end{block}

\end{frame}

\subsection{Binary lifting}

\begin{frame}[fragile]{\subsecname}
Jedyną wadą poprzedniego algorytmu było to, że skakaliśmy za każdym razem o tylko $1$
wierzchołek w górę. \pause Aby pozbyć się tego problemu, możemy skakać o kilka -- na przykład 
$2^k$ -- wierzchołków w górę.

\pause
\begin{block}{}
\begin{lstlisting}[language=C++]
  while (!jest_przodkiem(v, u))
    v = skocz_o_kilka(v);
  return v;
\end{lstlisting}
\end{block}

\pause
Tutaj pojawiają się $2$ problemy:
\begin{enumerate}
  \pause\item nie mamy zdefiniowanej funkcji do skakania;
  \pause\item nie kontrolujemy długości skoku (możemy skoczyć za daleko).
\end{enumerate}

\end{frame}

\begin{frame}[fragile]{\subsecname}
Jako jedno z rozwiązań uprzednio wspomnianych poprzednio problemów proponuję takie rozwiązanie.
Skaczemy wierzchołkiem $v$ do najniższego wierzchołka niebędącego przodkiem $u$.\\
Ojcem takiego wierzchołka będzie $LCA(v, u)$.\\
Jeśli chodzi o długość skoków, to będziemy skakać o coraz mniejsze potęgi dwójki -- można w ten sposób uzyskać dowolną liczbę.
\pause

\begin{block}{}
\begin{lstlisting}[language=C++]
  for (int skok = LOG_N; --skok >= 0;)
    if (!jest_przodkiem(ojciec[v][skok], u))
      v = ojciec[v][skok];
  return ojciec[v][0];
\end{lstlisting}
\end{block}

\end{frame}

\begin{frame}[fragile]{\subsecname}
Pozostaje nam wyznaczać szybko wartości funkcji \lstinline{ojciec}.\pause
Można łatwo ją spreprocesować wykonując \lstinline{dfs}-a z korzenia drzewa.

\begin{block}{}
\begin{lstlisting}[language=C++]
  int ojciec[MAX_N][LOG_N];
  void dfs(int v, int p) {
    visited[v] = true;
    ojciec[v][0] = p;
    for (int skok = 1; skok < LOG_N; skok++) {
      int u = ojciec[v][skok - 1];
      ojciec[v][skok] = ojciec[u][skok - 1];
    }
    for (auto u : G[v])
      if (!visited[u])
        dfs(u, v);
  }
\end{lstlisting}
\end{block}
\pause
Takie rozwiązanie okazuje się już być wystarczająco szybkie i pozwala na odpowiadanie na zapytania w złożoności $O(\log{n})$.

\end{frame}

\end{document}
