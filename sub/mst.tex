\documentclass[../main.tex]{subfiles}

\begin{document}

\section{Minimalne drzewo rozpinające (MST)}

\subsection{Wprowadzenie}

\begin{frame}{\subsecname}{Definicja problemu}

Niech $G$ będzie ważonym grafem nieskierowanym.\\
Chcemy znaleźć poddrzewo grafu $G$, które łączy wszystkie jego wierzchołki
(czyli jego drzewo rozpinające) i ma najmniejszą wagę (sumę wszystkich wag krawędzi) ze wszystkich
możliwych drzew rozpinających. Takie drzewo rozpinające nazywamy minimalnym drzewem rozpinającym (ang. \textit{minimum spanning tree}).

\end{frame}

\begin{frame}{\subsecname}{Przykład}
  
\begin{tikzpicture}

\begin{scope}[every node/.style={circle, draw, minimum size = 7.5mm, fill=gray!40}]
  \node[] (1) at (0,2) {1};
  \node[] (2) at (3,0) {2};
  \node[] (3) at (3,4) {3};
  \node[] (4) at (6,0) {4};
  \node[] (5) at (6,4) {5};
  \node[] (6) at (9,2) {6};
\end{scope}

\graph[edge = {draw, thick}] {
  (1) -- [edge label = 9, alt=<2>{transparent}] (3) 
      -- [edge label = 5, alt=<2>{transparent}] (5) 
      -- [edge label = 8, alt=<2>{transparent}] (6);
  (1) -- [edge label = 4]
  (2) -- [edge label = 2] (4)
      -- [edge label = 7] (6);
  (3) -- [edge label = 3, alt=<2>{transparent}] (4)
      -- [edge label = 3] (5);
  (3) -- [edge label = 1] (2);
};

\end{tikzpicture}

\end{frame}

\begin{frame}{\subsecname}{Właściwości}

\begin{itemize}
  \pause\item
    Jeżeli wszystkie wagi krawędzi są różne, to istnieje tylko jedno MST, w przeciwnym razie
    możliwe jest istnienie wielu.
  \pause\item
    MST jest także drzewem rozpinającym o najmniejszym iloczynie wag krawędzi
    (prosty dowód poprzez zamianę wag na ich logarytmy).
  \pause\item
    Największa waga krawędzi w MST jest najmniejsza spośród wszystkich maksymalnych wag
    krawędzi wszystkich możliwych drzew rozpinających.
  \pause\item
    Maksymalne drzewo rozpinające może zostać uzyskane tak samo jak MST, wystarczy 
    zamienić wagi krawędzi na ujemne.
\end{itemize}

\end{frame}

\subsection{Algorytm Kruskala}

\begin{frame}{\subsecname}

Twórcą algorytmu jest Joseph Bernard Kruskal. Algorytm jest bardzo prosty w działaniu:
\begin{enumerate}
  \pause\item sortujemy krawędzie niemalejąco według wag;
  \pause\item przeglądamy krawędzie od najlżejszych i zachłannie dodajemy do wyniku te, które łączą dwa do tej pory niepołączone wierzchołki.
\end{enumerate}

\end{frame}

\begin{frame}{\subsecname}{Przykład działania}
\begin{tikzpicture}

\begin{scope}[every node/.style={circle, draw, fill=gray!40}]
  \node[] (1) at (0,1) {1};
  \node[] (2) at (1,0) {2};
  \node[] (3) at (1,2) {3};
  \node[] (4) at (3,0) {4};
  \node[] (5) at (3,2) {5};
  \node[] (6) at (4,1) {6};

  \node[] (n1) at (6,1) {1};
  \node[] (n2) at (7,0) {2};
  \node[] (n3) at (7,2) {3};
  \node[] (n4) at (8,0) {4};
  \node[] (n5) at (8,2) {5};
  \node[] (n6) at (9,1) {6};
\end{scope}

\graph[edge = {draw, thick}] {
  (1) -- [edge label = 9, alt=<15>{red}] (3) 
      -- [edge label = 5, alt=<11>{red}] (5) 
      -- [edge label = 8, alt=<14>{red}] (6);
  (1) -- [edge label = 4, alt=<9>{red}] (2) 
      -- [edge label = 2, alt=<4>{red}] (4)
      -- [edge label = 7, alt=<12>{red}] (6);
  (3) -- [edge label = 3, alt=<8>{red}] (4)
      -- [edge label = 3, alt=<6>{red}] (5);
  (3) -- [edge label = 1, alt=<2>{red}] (2);
};

\only<3->\path[draw, thick] (n3) -- (n2);
\only<5->\path[draw, thick] (n2) -- (n4);
\only<7->\path[draw, thick] (n4) -- (n5);
\only<10->\path[draw, thick] (n1) -- (n2);
\only<13->\path[draw, thick] (n4) -- (n6);

\end{tikzpicture}
\end{frame}

\begin{frame}{\subsecname}{Dowód poprawności}

Niech $G$ będzie spójnym ważonym grafem, a $T$ będzie podgrafem $G$ wyprodukowanym, przez algorytm. 
\pause
$T$ nie może zawierać cyklu, gdyż z definicji wynika, że krawędzie tworzące cykl nie są dodawane. 
$T$ musi być spójne, gdyż każda napotykana krawędź łącząca dwie składowe $T$ zostaje do niego 
dodana. Zatem $T$ jest drzewem rozpinającym grafu $G$.\\
\pause
Dowód minimalności $T$ robi się indukcyjnie pokazując, że w każdym momencie algorytmu zbiór $F$
krawędzi $T$ zawiera się w zbiorze krawędzi pewnego z minimalnych drzew rozpinających grafu $G$.
Baza indukcyjna jest trywialna ($F = \varnothing$). 
\pause
Krok pozostawiam jako ćwiczenie.

\end{frame}

\begin{frame}[fragile]{\subsecname}

\begin{block}{}
\begin{lstlisting}[language = C++]
  struct krawedz {
    int v, u; long long w;
    bool operator<(krawedz e) { return w < e.w; }
  }; |\pause|
  ...
  sort(krawedzie.begin(), krawedzie.end());
  for (auto e : krawedzie) {
    if (polaczone(e.v, e.u)) continue;
    polacz(e.v, e.u);
    wynik.emplace_back(e);
  }
\end{lstlisting}
\end{block}

\end{frame}

\end{document}
