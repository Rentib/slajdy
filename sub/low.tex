\documentclass[../main.tex]{subfiles}

\begin{document}

\section{Parametr low (mosty i punkty artykulacji)}

\subsection{Wprowadzenie}

\begin{frame}{\subsecname}{Definicja problemu}

Dany jest nieskierowany graf $G = \langle V, E \rangle$.
Naszym zadaniem jest wyznaczenie wszystkich krawędzi, których usunięcie zwiększa liczbę
spójnych składowych $G$.

\end{frame}

\begin{frame}{\subsecname}{Przykład}

\begin{tikzpicture}
  
  \begin{scope}[every node/.style={circle, draw, minimum size = 7.5mm, fill=gray!40}]
    \node[] (1) at (4.1, 7.5) {1}; 
    \node[] (2) at (6.2, 8.0) {2}; 
    \node[] (3) at (5.8, 6.6) {3}; 
    \node[] (4) at (7.5, 6.7) {4}; 
    \node[] (5) at (7.2, 5.3) {5}; 
    \node[] (6) at (9.8, 7.3) {6}; 
    \node[] (7) at (10, 9.4) {7}; 
    \node[] (8) at (8.7, 10.9) {8}; 
    \node[] (9) at (7.1, 9.7) {9}; 
    \node[] (10) at (2.5, 5.9) {10}; 
    \node[] (11) at (1.6, 7.9) {11}; 
    \node[] (12) at (0.1, 8.2) {12}; 
    \node[] (13) at (0.0, 9.8) {13}; 
    \node[] (14) at (1.5, 9.3) {14}; 
  \end{scope}

  \graph[edge={draw, thick}] {
    (1) --[alt=<2>{red}] (2),
    (2) --[alt=<2>{red}] (3),
    (2) -- (4),
    (4) --[alt=<2>{red}] (5),
    (4) -- (6),
    (6) -- (7),
    (7) -- (8),
    (8) -- (9),
    (7) -- (9),
    (9) -- (2),
    (1) --[alt=<2>{red}] (10),
    (13) -- (14),
    (14) -- (11),
    (10) --[alt=<2>{red}] (11),
    (11) -- (12),
    (12) -- (13),
  };

\end{tikzpicture}

\end{frame}

\subsection{Algorytm}

\begin{frame}{\subsecname}

  Wybierzmy dowolny wierzchołek \textit{root} i odpalmy na nim \textit{DFS}-a.
  Łatwo jest udowodnić następujący fakt:

  \begin{itemize}
    \pause\item
      Powiedzmy, że jesteśmy w wierzchołku $v$. Krawędź $(v, u)$, gdzie $u$ jest synem $v$ w
      drzewie DFS, jest mostem wtedy i tylko wtedy, gdy żaden z wierzchołków poddrzewa DFS
      wierzchołka $u$ nie ma krawędzi wstecznej do $v$ lub jego przodka w drzewie DFS.
  \end{itemize}

\end{frame}

\begin{frame}{\subsecname}

  Oznaczmy czas wejścia (preorder) wierzchołka $v$ poprzez \textit{tin}$[v]$.\\
  \pause\[
    \textit{low}[v] = \min
    \begin{cases}
      \textit{tin}[v]\\
      \textit{tin}[p] &\forall\text{$p$ takich, że krawędź $(v,p)$ jest wsteczna}\\
      \textit{low}[u] &\forall\text{$u$ takich, że krawędź $(v,u)$ jest w drzewie}\\
    \end{cases}
  \]\pause
  Przy tak zdefiniowanej funkcji \textit{low} krawędź $(v, u)$ w drzewie DFS jest mostem wtedy
  i tylko wtedy, gdy $\textit{tin}[v] < \textit{low}[u]$.\\

  Złożoność algorytmu jest oczywiście taka jak złożoność \textit{DFS}-a, czyli $O(n + m)$.

\end{frame}

\begin{frame}[fragile]{\subsecname}{Implementacja}

\begin{block}{}
\begin{lstlisting}[language = C++]
  vector<int> G[MAX_N];
  bool visited[MAX_N];
  int low[MAX_N], tin[MAX_N], timer = 0;
  void dfs(int v, int p = -1) {
    visited[v] = true;
    tin[v] = low[v] = ++timer;
    for (auto u : G[v]) {
      if (u == p) continue;
      if (visited[u]) {
        low[v] = min(low[v], tin[u]);
      } else {
        dfs(u, v);
        low[v] = min(low[v], low[u]);
        if (tin[v] < low[u]) JEST_MOSTEM(v, u);
      }
    }
  }
\end{lstlisting}
\end{block}

\end{frame}

\end{document}
